\chapter{Conclusion}
\setlength{\parskip}{2.5ex plus .4ex minus .4ex}
%1. Ce que le stage m'a apporté
%2. Logiciel fonctionnel..
%Ce qui reste, tests fonctionnel, de performance
The two first months of internship make me see a research environment: discovering a new project and finding a subject. After that I could design a solution by myself and develop it.\\
I learnt a lot with the autonomy I had: designing, implementing, testing and finding a solution to the bugs who occured, but also technical skills like C++.\\

The interface I developped is functionnal. Indeed, ROS users can now use my package or may be SIGVerse users who thinks using ROS is an easier way to use SIGVerse. This objective of achievement has been very motivated.\\
However few improvements can be made like the complete integration of SLAM or adding new interaction with the inverse kinematics package.\\

This internship took place in Japan that gave me more than technical skills, a knowledge of a very different culture. It is a new point of view that will be very useful for a potential next work abroad or just having a better open mind, ideas. 


%Grâce à ce stage, j'ai pu voir comment développer de nouvelles fonctionnalités à partir de besoins précisés en début de stage. Ces besoins s'attachaient à un logiciel existant, il a donc fallu apprendre à lire et comprendre le fonctionnement avant toute modification et/ou ajout de code.\\
%J'ai également pu voir plusieurs aspects de développement, créer un plugin sur un logiciel existant, créer les fonctionnalités de ce plugin, créer son interface graphique mais aussi l'aspect debuguage, rapport de bug du logiciel existant et debuguage de mon propre code.\\
%Ce stage m'a apporté des compétences telles que le moyen de répondre à un besoin, de rechercher des informations, mais aussi des compétences techniques notamment en C++.\\
%
%A la fin de ce stage, le plugin développé est fonctionnel, les usagés peuvent s'en servir afin de modéliser des individus dans le sens IBM.\\
%Ce plugin fournit de nouvelles possibilités que VLE ne pouvait pas offrir auparavant. Cependant, du travail reste à faire comme des tests sur les fonctionnalités. En effet, le plugin répond actuellement aux besoins cités en première partie mais peut-être que de nouveaux cas d'utilisation sont à venir et de nouvelles fonctionnalités seraient nécessaires comme par exemple faciliter l'inclusion de modèle IBM dans d'autres modèles, proposer à l'utilisateur une interface afin qu'il choisisse les ports d'entrée et de sortie qui lui sont nécessaire...\\
%De plus, certaines fonctionnalités comme l'observation de variable globale est possible mais aucune interface graphique n'a été développée.\\
